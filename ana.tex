\documentclass[a4paper,8pt]{extarticle}
\usepackage{amssymb,amsmath,amsthm,amsfonts}
\usepackage{multicol,multirow}
\usepackage{calc}
\usepackage{ifthen}
\usepackage{tabularx}
\usepackage[utf8]{inputenc}
\usepackage[landscape]{geometry}
\usepackage[colorlinks=true,citecolor=blue,linkcolor=blue]{hyperref}
\usepackage{accents}
\newcommand{\vect}[1]{\accentset{\rightharpoonup}{#1}}

\ifthenelse{\lengthtest { \paperwidth = 11in}}
    { \geometry{top=.5in,left=.5in,right=.5in,bottom=.5in} }
	{\ifthenelse{ \lengthtest{ \paperwidth = 297mm}}
		{\geometry{top=1cm,left=1cm,right=1cm,bottom=1cm} }
		{\geometry{top=1cm,left=1cm,right=1cm,bottom=1cm} }
	}
\pagestyle{empty}
\makeatletter
\renewcommand{\section}{\@startsection{section}{1}{0mm}%
                                {-1ex plus -.5ex minus -.2ex}%
                                {0.5ex plus .2ex}%x
                                {\normalfont\large\bfseries}}
\renewcommand{\subsection}{\@startsection{subsection}{2}{0mm}%
                                {-1explus -.5ex minus -.2ex}%
                                {0.5ex plus .2ex}%
                                {\normalfont\normalsize\bfseries}}
\renewcommand{\subsubsection}{\@startsection{subsubsection}{3}{0mm}%
                                {-1ex plus -.5ex minus -.2ex}%
                                {1ex plus .2ex}%
                                {\normalfont\small\bfseries}}
\makeatother
\setcounter{secnumdepth}{0}
\setlength{\parindent}{0pt}
\setlength{\parskip}{0pt plus 0.5ex}
% -----------------------------------------------------------------------

\title{Analiza 2}

\begin{document}

\raggedright
\footnotesize

\begin{multicols}{4}
\setlength{\premulticols}{1pt}
\setlength{\postmulticols}{1pt}
\setlength{\multicolsep}{1pt}
\setlength{\columnsep}{2pt}

\section{Uporaba integrala}

\textbf{Računanje površine ravninskih likov pod krivulijo}
\[
    p = \int^b_a f(x)\ dx
\]

\textbf{Dolžina ravninske krivulije}
\[ s = \int^b_a \sqrt{1+f'(x)^2}\ dx \]

\textbf{Prostornina in površina vrtenine} (vrtimo okoli $x$ osi)
\[ V = \pi \int^b_a f(x)^2\ dx\]
\[ P = 2\pi \int^b_a f(x)\sqrt{1+(f'(x))^2}\ dx\]

\textbf{Prostornina vrtenine} (vrtimo okoli $y$ osi)
\[ V = 2\pi \int^b_a x f(x)\ dx \]

\textbf{Težišče ravninskih likov}
\[ y_T = \frac{\int^b_a f(x)^2 \ dx}{2p}\]

Doložina poti, ki jo pri vrtenju za $360^\circ$ opiše težišče je $2\pi y_T$.
\[ 2\pi y_T p = \pi \int^b_a f(x)^2 \ dx = V\]


\textbf{Težišče ravninske krivulije}
\[ y_T = \frac{\int^b_a f(x) \sqrt{1+(f'(x))^2} \ dx}{s}\]

\textbf{Dolžina ravninske krivulije}
\[ y_T = \frac{\int^b_a f(x) \sqrt{1+(f'(x))^2}\ dx}{s} \]

\section{Parametrično podane krivulje}
Enačba krivulje v $\mathbb{R}^3$ je oblike
\[ \vect{r} : [a,b] \to \mathbb{R}^3 \]
\[ \vect{r}(t) = (x(t), y(t), z(t)) \in \mathbb{R}^3 \]

\subsubsection{Krivulje v polarnih koordinatah}
\[r = r(\varphi)\]
\[x(\varphi) = r(\varphi) \cos \varphi\]
\[y(\varphi) = r(\varphi) \sin \varphi\]
\[ \vect{r} (\varphi) = ( r(\varphi) \cos \varphi, r(\varphi) \sin \varphi)\]

\subsubsection{Tangenta parametrične krivulije}
Tangenta je kar vektor, ki ga dobimo z odvajanjem parametrične enačbe.
\[ \vect{r}(t) = (x(t), y(t), z(t))\]
\[ \dot{\vect{r}}(t) = (\dot{x}(t), \dot{y}(t), \dot{z}(t))\]

\subsubsection{Doložina parametrično podane krivulije}
Majhen delček krivulije ima dolžino $ds = \| \vect{r}(t+dt) - \vect{r}(t)\| = \|\dot{\vect{r}}(t)\| $. 
Doložina večjega dela krivulje je potem integral teh delčkov:
\[s = \int^b_a \|\dot{\vect{r}}(t)\|\ dt\]

\subsubsection{Doložina krivulije v polarnih koordinatah}
\[s = \int^b_a \sqrt{\dot{r}(\varphi)^2 + r(\varphi)^2}\ d\varphi\]

\subsubsection{Naravna parametrizacija}
Pot $\vect{r} : I \to \mathbb{R}^3$ je naravno parametrizirana, če je $\forall t\in I\ :\ |\dot{\vect{r}}(t)| = 1$.\\
Vsako pot lahko \textbf{naravno reparametriziramo}. Za nek $a\in I$ definiramo funkcijo
\begin{equation*}
    \begin{aligned}
        s:\ & I \to J & s(t) =& \int_a^t |\dot{\vect{r}}(\tau)|d\tau\\
    \end{aligned}
\end{equation*}
z $t : J \to I$ označimo izverz od s $s$. Potem je $\vect{\varphi}(s) : J \to \mathbb{R}$ dana s predpisom $\vect{\varphi}(s) = \vect{r}(t(s))$ \emph{naravno parametrizirana pot}.

\subsubsection{Frenetove formule}
Naj bo $\vect{r}(s)$ naravna parametrizacija.
\begin{equation*}
    \begin{aligned}
        \underbrace{\vect{T} = \vect{r}\,'}_\textmd{tangenta} &&
        \underbrace{\vect{N} = \frac{\vect{T}'}{|\vect{T}'|}}_\textmd{normala} &&
        \underbrace{\vect{B} = \vect{T} \times \vect{N}}_\textmd{binormala}
    \end{aligned}
\end{equation*}
$\kappa = |\vect{T}'| = |\vect{r}\,''|\ \dots $ fleksijska ukrivljenost\\\ \\
$\tau = \frac{(\vect{r}\,' \times \vect{r}\,'')\vect{r}\,'''}{|\vect{r}\,''|^2}\ \dots $ torzijska ukrivljenost

Če je $\forall s \in I\ :\ \kappa(s) \neq 0$ so vektorji $\vect{T},\vect{N},\vect{B}$ dobro definirani in veljajo \textbf{Frenetove formule}:
\begin{equation*}
    \begin{aligned}
        \vect{T}' = \kappa \vect{N}\quad &&
        \vect{N}' = \tau\vect{B}-\kappa \vect{T} && \quad
        \vect{B}' = -\tau \vect{N}
    \end{aligned}
\end{equation*}

\subsubsection{Frenetova baza in ukrivljenost v poljubni parametrizaciji}
Naj bo $\vect{r}(t)$ poljubna regularna pot z neničelno fleksijska ukrivljenostjo.
\begin{equation*}
    \begin{aligned}
        \vect{T} = \frac{\dot{\vect{r}}}{|\dot{\vect{r}}|} \quad &&
        \vect{B} = \frac{\dot{\vect{r}} \times \ddot{\vect{r}}}{|\dot{\vect{r}} \times \ddot{\vect{r}}|} && \quad
        \vect{N} = \vect{B} \times \vect{T}\\
    \end{aligned}
\end{equation*}
\begin{equation*}
    \begin{aligned}
        \kappa = \frac{|\,\dot{\vect{r}} \times \ddot{\vect{r}}\,|}{|\,\dot{\vect{r}}\,|^3} && \quad
        \tau = \frac{\big(\,\dot{\vect{r}} \times \ddot{\vect{r}}\,\big)\dddot{\vect{r}}}{|\,\dot{\vect{r}} \times \ddot{\vect{r}}\,|^2}
    \end{aligned}
\end{equation*}

\section{Metrični prostor}
Preslikava $d:M\to M$ je \textbf{metrika} na množici $M$, če velja:
\begin{itemize}
    \item $d(x,y) \geq 0$ in $d(x,y) = 0 \Leftrightarrow x = y$
    \item $d(x,y) = d(y,x)$
    \item $d(x,z) \leq d(x,y) + d(y,z)$
\end{itemize}
\textbf{Metrični prostor} je par $(M,d)$.\\
\textbf{Normiran prostor} je \emph{vektorski prostor} $V$ opremljen z normo.
\textbf{Norma} je preslikava $\|\,\| : V \to \mathbb{R}^+$, ki ima lastnosti:
\begin{itemize}
    \item $\|v\| = 0 \Rightarrow v = 0$
    \item $\|\alpha v \| = |\alpha| \|v\|$
    \item $ \| w + v \| \leq \| w \| + \| v \| $
\end{itemize}
Potem je $V$ tudi metrični prostor z $d(w,v) = \| w-v \|$.
\subsubsection{Odprte in zaprte množice}
\textbf{Odprta krogla} $B(a,r) = \left\{ x \in M : d(a,x) < r \right\}$\\
\textbf{Zaprta krogla} $\overline{B}(a,r) = \left\{ x \in M : d(a,x) \leq r \right\}$\\
Množiva $U \subseteq M$ je \textbf{odprta}, če 
\[\forall a \in U\ \exists \varepsilon > 0 : B(a,\varepsilon) \subseteq U \]
Množica $U \subset M$ je \textbf{zaprta}, če je $U^C$ odprta.\\
\textbf{Zaprtje} $\bar{A}$ množice $A$ je najmanjša množica v $M$, ki vsebuje $A$. \\
\textbf{Notranjost} $\mathring{A}$ je največja odprta množiva vsebovana v $A$.\\
\subsubsection{Rob množice}
Naj bo $A \subseteq M$.
Točka $a\in A$ je \textbf{robna}, če vsaka odprta krogla $B(a,\varepsilon)$ seka $A$ in $A^C$. \\
\textbf{Rob} $ \partial A $ je množica vseh robnih točk množice $A$.
\subsubsection{Zaporedja}
Zaporedje ($a_n$) z limito v metričenem porstoru ($(M,d)$) je \textbf{konvergentno} z limito $a \in M$, če velja:
\[ \lim_{n \to \infty} d(a_n, a) = 0 \]
\[ \forall \varepsilon > 0\ \exists n_0 \in \mathbb{N} : n \geq n_0 \Rightarrow d(a_n, a) < \varepsilon \]
Zaporedje ($a_n$) je \textbf{Cauchyevo}, če velja:
\[ \forall \varepsilon > 0\ \exists n_0 \in \mathbb{N} : n,m \geq n_0 \Rightarrow d(a_n, a_m) < \varepsilon \]
Vsako konvergentno zaporedje je tudi Cauchyevo. Obratno pa je res samo, če je metrični prostor \textbf{poln}. ($\mathbb{R}^n$ je poln)
\subsubsection{Kompaktnost}
\subsubsection{Zveznost}
\subsubsection{Skrčitve}
\subsubsection{Banachovo načelo}
\end{multicols}
\end{document}